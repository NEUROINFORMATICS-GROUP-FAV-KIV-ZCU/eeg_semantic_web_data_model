\documentclass{projekt}
\usepackage[none]{hyphenat}
\usepackage{url}
\usepackage[utf8]{inputenc}
\usepackage[T1]{fontenc} 
\usepackage{ae}
\usepackage{fancyhdr}
\usepackage{graphicx}
\usepackage{pdfpages}
\usepackage[all]{xy}
\usepackage[czech]{babel}
\usepackage[activate={true,nocompatibility,all}, stretch=10, shrink=10, step=1, auto=true, draft=false]{microtype}
%\usepackage[total={17cm,28cm}, top=3cm, left=2cm, includefoot]{geometry}
\frenchspacing
\author{Filip Markvart}
\title{Datový model EEG/ERP portálu v prostředcích sémantického webu}
\titlet{} 
\titlett{}
\university{Západočeská univerzita v Plzni}
\faculty{Fakulta aplikovaných věd}
\department{Katedra informatiky a výpočetní techniky}
\subject{Diplomová práce}
\town{Plzeň}
\begin{document}
\pagestyle{fancy}
\renewcommand{\chaptermark}[1]{\markboth{\textit{#1}}{}}
\renewcommand{\sectionmark}[1]{\markright{\textit{#1}}{}}
\cfoot{\thepage}
\lhead{\leftmark}
\rhead{\rightmark}
\maketitle
\chapter*{Prohlášení}
\thispagestyle{empty}
Prohlašuji, že jsem diplomovou práci vypracoval samostatně a výhradně s~použitím citovaných pramenů.
\vskip 1.5em
V Plzni dne \today
\vskip 0.7em
\hskip 9cm Filip Markvart
\chapter*{Abstract}
\thispagestyle{empty}
\hspace{0.65cm}This thesis DODELAT
%describes a transformation of neuroinformatics metadata stored in the relational database to the semantic web standard. The main goal is to investigate existing tools and necessary %modifications that ensure automatic transformation for converting input data. The modified tool is able to process additional semantic information that cannot be stored in the relational %database.


%The last part describes design and implementation of an application that uses annotations to add semantic information to transformed data.

\tableofcontents
\pagestyle{fancy}
\renewcommand{\chaptermark}[1]{\markboth{\textit{#1}}{}}
\renewcommand{\sectionmark}[1]{\markright{\textit{#1}}{}}
\cfoot{\thepage}
\lhead{\leftmark}
\rhead{\rightmark}
\parskip 1em
\chapter{Úvod}
\hspace{0.65cm}EEG/ERP portál je webová aplikace sloužící výzkumným pracovníkům ke shromažďování a organizaci dat získaných při neuroinformatických experimentech v EEG laboratoři. Jejím cílem je ukládání naměřených dat v kontextu prováděného experimentu, který lze popsat rozsáhlou množinou různorodých údajů. Tato aplikace již prošla mnohaletým vývojem v jehož průběhu postupně docházelo ke změnám datového modelu kvůli přibývajícím požadavkům na uchovávaná data. Relační databáze jež slouží jako persistentní úložiště tak postupně byla rozšiřována o další tabulky, jejichž počet se k datu tvorby této práce pohybuje v řádu desítek. Většina realizací požadavků na ukládání dalších dat tak přímo znamená zásah nejen do databáze portálu ale také to do datové vrstvy, která ji využívá. V současné době tak databáze obsahuje velké množství tabulek uchovávající různá data, která jsou ale ve smyslu sémantiky často příbuzná a existuje mezi nimi vazba, která je prostřednictvím relačního datového modelu velmi obtížně popsatelná. Zároveň lze očekávat, že budou přibývat požadavky na uchování dalších dat, která navíc nemusejí mít jen homogenní strukturu (ve smyslu relační databáze), ale může se jednat i o množiny sémanticky příbuzných údajů – tzv. metadata, které budou vázány pouze k některým datům. Možnost ukládání strukturně heterogenních, ale sémantický příbuzných metadat je tak dalším otevřeným problémem.

Cílem této práce je prozkoumání struktury a nalezení sémantiky dat v současném datovém modelu relační databáze portálu a následná úprava tohoto modelu do podoby, která by dovolovala uchovat jak sémantiku dat, kterou není možné relačním modelem vyjádřit tak dodávat dynamicky datům přídavná metadata, aniž by muselo docházet k větším zásahům do datového modelu portálu. Pro realizaci úpravy datového modelu budou v této práci využity prostředky tzv. sémantického webu, který poskytuje množství standardů a technologií pro uchovávání organizaci a správu dat. Tyto technologie a nástroje zde budou popsány a na základě jejich analýzy budou vybrány prostředky, které se využijí pro implementaci úpravy zmiňovaného datového modelu. 
Poslední část práce se věnuje testování modifikovaného modelu a to především z výkonnostního hlediska. Díky této části by mělo být možné posoudit jak užitečnost samotné úpravy tak i použitelnost a efektivnost získaného modelu pro potřeby EEG/ERP portálu.


\end{document}
